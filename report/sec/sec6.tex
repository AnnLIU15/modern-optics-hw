\section{总结}

由于自身并非做光学相关的领域,研究领域也与光没有很直接的联系,对光学的了解甚微,出于好奇选择了这门课。光学基础薄弱一时间难以明白前几个题目的思路,因此选择这个不需要过多的光学知识与背景的题目作为期末作业。

本题目的四个子目标本文都基本完成,对有了色度学更深的认知与理解,明白了成色的原理以及一些色度的概念与定义,在第三个子目标中分析较为欠缺,第四个子目标中完成度较低,设置较为粗糙。总体来说达到了自身满意的水平,毕竟在开始做的时候只知道RGB与存在混色原理,对其他东西一窍不通,最后做出了一个差强人意的结果,感觉已经达到预期水准。最后吐槽以下,RGB三基色LED灯珠的数据真的太难找了,本次作业的一半的有效工作时间都在找数据与找数据的路上,由于相关知识的欠缺与渠道的阻塞,费劲千辛万苦才找到了原始数据,虽然不清楚其是否标准,但起码比从图像中扣要合理,曾几何时我想过从论文中的QLED图扣下来,后面也因为找到勉强能用的数据才就此罢休。本文主要使用了宾夕法尼亚州立大学llab的数据,也从中获取到了很多xyz计算的思路。本文中部分图是jupyter(ipynb)中保存的,可能py后缀的文件中没有直接对应的功能,需要部分的修改才能重新。

本次大作业让我学会了很多的色度学知识,虽然这些知识对我未来的研究学习可能并没有很明显的帮助,但是本次大作业拓展了我对色度、色域相关的认知,对我日后生活有较大的帮助,因为可以更懂得如何评价一个显示器、设备的好坏,不易买到不合适的产品。最后,感谢老师们的教导,让我对光学了解更多了。虽然由于前置知识的欠缺大部分内容听不太懂,但也增广了见识。